\documentclass[10pt]{exam}
\usepackage[T1]{fontenc}
\usepackage[paper=a4paper,margin=2cm]{geometry}
\usepackage[sfdefault,light]{roboto}

\usepackage[usenames,dvipsnames]{xcolor}
\usepackage{amsmath,amssymb,graphicx,enumitem,listings,multicol,titlesec}
\usepackage{mathtools}

\setlength\parindent{0cm}

\titleformat{\section}{\normalfont\Large\bfseries}{}{0em}{}[{\titlerule[0.5pt]}]
\titleformat{\subsection}{\normalfont\large\bfseries}{}{0em}{}

\newcommand{\floor}[1]{\left\lfloor #1 \right\rfloor}
\newcommand{\code}[1]{\lstinline{#1}}
\newcommand{\blankpage}{\null\thispagestyle{empty}\addtocounter{page}{-1}\newpage}

\lstset{language=Java,
	basicstyle=\ttfamily,
	commentstyle=\color{black!45},
	keywordstyle=\bfseries,
	showstringspaces=false}

\newcommand{\EBox}[2]{
	\colorbox{black!10}{\parbox{0.9875\textwidth}{\bfseries
		#1
	}}

	\colorbox{black!5}{\parbox{0.9875\textwidth}{
		#2
	}}
}

\newcounter{QuestionCounter}
\newcommand{\QBox}[2]{
	\stepcounter{QuestionCounter}
	\colorbox{black!10}{\parbox{0.9875\textwidth}{
		\textbf{Question \#\theQuestionCounter:} #1
	}}

	\colorbox{black!5}{\parbox{0.9875\textwidth}{
		\ \\[#2]
	}}
}

\header{\footnotesize\scshape Lab \#\LabNumber: \LabTitle}{}{}
\cfoot{\footnotesize\scshape \LabCourse\\Woodstock School---Mussoorie, Uttarakhand---India}


%Color-Coded questions
\newcommand{\ColourQuestion}[3]{\renewcommand{\questionlabel}{\colorbox{#1}{\bfseries\color{white}\thequestion}\hfill}\question[#3] #2}
\newcommand{\BlueQuestion}[1]{\ColourQuestion{RoyalBlue}{#1}{3}}
\newcommand{\GreenQuestion}[1]{\ColourQuestion{ForestGreen}{#1}{5}}
\newcommand{\YellowQuestion}[1]{\ColourQuestion{Goldenrod}{#1}{10}}
\newcommand{\RedQuestion}[1]{\ColourQuestion{BrickRed}{#1}{20}}
\newcommand{\PurpleQuestion}[1]{\ColourQuestion{RoyalPurple}{#1}{30}}
\pointsinrightmargin

\header{\footnotesize\scshape Assignment \#\AssignmentNumber: \AssignmentTitle}{}{}
\cfoot{\footnotesize\scshape \AssignmentCourse\\Woodstock School---Mussoorie, Uttarakhand---India}


\def\AssignmentCourse{AP Computer Science A}
\def\AssignmentNumber{00}
\def\AssignmentTitle{Sample Assignment}

\begin{document}
  \begin{questions}
    \BlueQuestion{Add a method, \code{peek()}, to \code{Stack} that returns the most recently inserted element on the stack, without removing it.}

    \BlueQuestion{Add a method, \code{isEmpty()}, to \code{Stack} that returns \code{true} if there are no elements in the stack and \code{false} otherwise.}

    \GreenQuestion{Suppose that a client performs an intermixed sequence of \code{push()} and \code{pop()} operations on a stack of integers. The push operations put the integers $0$ through $9$, in order, on the stack; the pop operations print out the return value. Which of the following sequence(s) could \emph{not} occur?}
      \begin{multicols}{2}
        \begin{enumerate}[label=\texttt{\alph*.}]
          \item 4 3 2 1 0 9 8 7 6 5
          \item 4 6 8 7 5 3 2 9 0 1
          \item 2 5 6 7 4 8 9 3 1 0
          \item 4 3 2 1 0 5 6 7 8 9
          \item 1 2 3 4 5 6 9 8 7 0
          \item 0 4 6 5 3 8 1 7 2 9
          \item 1 4 7 9 8 6 5 3 0 2
          \item 2 1 4 3 6 5 8 7 9 0
        \end{enumerate}
      \end{multicols}

    \GreenQuestion{Suppose \code{stack} is an object representing an implemented stack of integers. What does the following code fragment print when \code{N} is $50$? Explain what your output represents (you might want to try additional values for \code{N} to verify your answer).}
      \begin{lstlisting}
        while (N > 0) {
          stack.push(N % 2);
          N = N / 2;
        }
        while (!stack.isEmpty()) {
          System.out.print(stack.pop())
        }
        System.out.println()
      \end{lstlisting}
    \YellowQuestion{Write the method \code{Parentheses()} that takes a string of different parentheses and uses a stack to determine whether its parenthesis are properly balanced. For example, your method should return \code{true} for\\ ``\texttt{[()]\{[()()]()\}}'' and \code{false} for ``\texttt{[(])}''. }\\[4pt]
    {\small\textbf{Hint:} Use the \code{substring()} method of the \code{String} class.}

    \RedQuestion{Suppose that we have a sequence of intermixed \code{push()} and \code{pop()} operations on a stack of integers. The push operations put the $N$ integers, $0$ through $N - 1$, in order on the stack; the pop operations print out the return value. Write the method, \code{ValidPermutation()}, that will return \code{true} if a given string representing a space-separated sequence of these integers represents valid output for this stack and \code{false} otherwise.}\\[4pt]
    {\small\textbf{Hint \#1:} Your method should also take \code{N} as a parameter.}\\
    {\small\textbf{Hint \#2:} This method should be able to answer Question \#3 for you.}\\
    {\small\textbf{Hint \#3:} The \code{split()} method of the \code{String} class may be useful.}
  \end{questions}
\end{document}
