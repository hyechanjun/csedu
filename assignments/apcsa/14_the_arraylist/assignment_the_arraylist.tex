\documentclass[10pt]{exam}
\usepackage[T1]{fontenc}
\usepackage[paper=a4paper,margin=2cm]{geometry}
\usepackage[sfdefault,light]{roboto}

\usepackage[usenames,dvipsnames]{xcolor}
\usepackage{amsmath,amssymb,graphicx,enumitem,listings,multicol,titlesec}
\usepackage{mathtools}

\setlength\parindent{0cm}

\titleformat{\section}{\normalfont\Large\bfseries}{}{0em}{}[{\titlerule[0.5pt]}]
\titleformat{\subsection}{\normalfont\large\bfseries}{}{0em}{}

\newcommand{\floor}[1]{\left\lfloor #1 \right\rfloor}
\newcommand{\code}[1]{\lstinline{#1}}
\newcommand{\blankpage}{\null\thispagestyle{empty}\addtocounter{page}{-1}\newpage}

\lstset{language=Java,
	basicstyle=\ttfamily,
	commentstyle=\color{black!45},
	keywordstyle=\bfseries,
	showstringspaces=false}

\newcommand{\EBox}[2]{
	\colorbox{black!10}{\parbox{0.9875\textwidth}{\bfseries
		#1
	}}

	\colorbox{black!5}{\parbox{0.9875\textwidth}{
		#2
	}}
}

\newcounter{QuestionCounter}
\newcommand{\QBox}[2]{
	\stepcounter{QuestionCounter}
	\colorbox{black!10}{\parbox{0.9875\textwidth}{
		\textbf{Question \#\theQuestionCounter:} #1
	}}

	\colorbox{black!5}{\parbox{0.9875\textwidth}{
		\ \\[#2]
	}}
}

\header{\footnotesize\scshape Lab \#\LabNumber: \LabTitle}{}{}
\cfoot{\footnotesize\scshape \LabCourse\\Woodstock School---Mussoorie, Uttarakhand---India}


%Color-Coded questions
\newcommand{\ColourQuestion}[3]{\renewcommand{\questionlabel}{\colorbox{#1}{\bfseries\color{white}\thequestion}\hfill}\question[#3] #2}
\newcommand{\BlueQuestion}[1]{\ColourQuestion{RoyalBlue}{#1}{3}}
\newcommand{\GreenQuestion}[1]{\ColourQuestion{ForestGreen}{#1}{5}}
\newcommand{\YellowQuestion}[1]{\ColourQuestion{Goldenrod}{#1}{10}}
\newcommand{\RedQuestion}[1]{\ColourQuestion{BrickRed}{#1}{20}}
\newcommand{\PurpleQuestion}[1]{\ColourQuestion{RoyalPurple}{#1}{30}}
\pointsinrightmargin

\header{\footnotesize\scshape Assignment \#\AssignmentNumber: \AssignmentTitle}{}{}
\cfoot{\footnotesize\scshape \AssignmentCourse\\Woodstock School---Mussoorie, Uttarakhand---India}


\def\AssignmentCourse{AP Computer Science A}
\def\AssignmentNumber{00}
\def\AssignmentTitle{Sample Assignment}

\begin{document}
  \begin{questions}
    \BlueQuestion{Explain why the following code fragment does not work as intended.}
      \begin{lstlisting}
        ArrayList names = new ArrayList();
        names.add("john");
        names.add("greg");
        ArrayList initials = new ArrayList();
        for (int i = 0; i < names.size(); i++)
          initials.add(names.get(i).substring(0, 1));
      \end{lstlisting}
      
    \BlueQuestion{Write a method that takes an \code{ArrayList<String>} as its sole parameter and returns a new \code{ArrayList<String>} in which the elements from the given \code{ArrayList} are stored in reverse order. Your method should not change the original list.}

    \GreenQuestion{Write a method that removes the smallest value from a given \code{ArrayList<Integer>}.}\\
      {\small\textbf{Hint:} The \code{Integer} class has a \code{compareTo()} method.}

    \GreenQuestion{Can an \code{ArrayList<Object>} be an element of itself? Test this hypothesis and explain your results.}

    \YellowQuestion{The following program code is designed to remove all instances of the word ``hello'' from a populated \code{ArrayList} of \code{Strings} called \code{words}.}
      \begin{lstlisting}
        int i = 0;
        while (i < words.size()) {
          if ("hello".compareTo(words.get(i)) == 0)
            words.remove(i);
          i++;
        }
      \end{lstlisting}
      \begin{parts}
        \part Explain why the above code fragment does not work as intended.
        \part Without changing any of the existing lines of code, add or insert a single line to enable the code fragment to work as intended.
      \end{parts}

    \RedQuestion{\emph{Failing Silently.} A method or program ``failing silently'' means that it fails to do what it was designed to without producing an error or any other message.}
      \begin{parts}
        \part Create the subclass of \code{ArrayList}, \code{SilentArrayList}, that will override each method of \code{ArrayList} that requires an \code{index} value and fails silently if that value is not a valid index of the \code{ArrayList}.\\
        {\small\textbf{Hint:} Remember that the keyword \code{super} can be used to access methods of the super class.}
        \part Describe a situation where failing silently might be the desired outcome for a method or program.
      \end{parts}
  \end{questions}
\end{document}
