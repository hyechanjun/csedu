\documentclass[10pt]{exam}
\usepackage[T1]{fontenc}
\usepackage[paper=a4paper,margin=2cm]{geometry}
\usepackage[sfdefault,light]{roboto}

\usepackage[usenames,dvipsnames]{xcolor}
\usepackage{amsmath,amssymb,graphicx,enumitem,listings,multicol,titlesec}
\usepackage{mathtools}

\setlength\parindent{0cm}

\titleformat{\section}{\normalfont\Large\bfseries}{}{0em}{}[{\titlerule[0.5pt]}]
\titleformat{\subsection}{\normalfont\large\bfseries}{}{0em}{}

\newcommand{\floor}[1]{\left\lfloor #1 \right\rfloor}
\newcommand{\code}[1]{\lstinline{#1}}
\newcommand{\blankpage}{\null\thispagestyle{empty}\addtocounter{page}{-1}\newpage}

\lstset{language=Java,
	basicstyle=\ttfamily,
	commentstyle=\color{black!45},
	keywordstyle=\bfseries,
	showstringspaces=false}

\newcommand{\EBox}[2]{
	\colorbox{black!10}{\parbox{0.9875\textwidth}{\bfseries
		#1
	}}

	\colorbox{black!5}{\parbox{0.9875\textwidth}{
		#2
	}}
}

\newcounter{QuestionCounter}
\newcommand{\QBox}[2]{
	\stepcounter{QuestionCounter}
	\colorbox{black!10}{\parbox{0.9875\textwidth}{
		\textbf{Question \#\theQuestionCounter:} #1
	}}

	\colorbox{black!5}{\parbox{0.9875\textwidth}{
		\ \\[#2]
	}}
}

\header{\footnotesize\scshape Lab \#\LabNumber: \LabTitle}{}{}
\cfoot{\footnotesize\scshape \LabCourse\\Woodstock School---Mussoorie, Uttarakhand---India}


%Color-Coded questions
\newcommand{\ColourQuestion}[3]{\renewcommand{\questionlabel}{\colorbox{#1}{\bfseries\color{white}\thequestion}\hfill}\question[#3] #2}
\newcommand{\BlueQuestion}[1]{\ColourQuestion{RoyalBlue}{#1}{3}}
\newcommand{\GreenQuestion}[1]{\ColourQuestion{ForestGreen}{#1}{5}}
\newcommand{\YellowQuestion}[1]{\ColourQuestion{Goldenrod}{#1}{10}}
\newcommand{\RedQuestion}[1]{\ColourQuestion{BrickRed}{#1}{20}}
\newcommand{\PurpleQuestion}[1]{\ColourQuestion{RoyalPurple}{#1}{30}}
\pointsinrightmargin

\header{\footnotesize\scshape Assignment \#\AssignmentNumber: \AssignmentTitle}{}{}
\cfoot{\footnotesize\scshape \AssignmentCourse\\Woodstock School---Mussoorie, Uttarakhand---India}


\def\AssignmentCourse{AP Computer Science A}
\def\AssignmentNumber{08}
\def\AssignmentTitle{Class Inheritance}

\begin{document}
  \begin{questions}
    \BlueQuestion{Indicate whether each of the following statements is \code{true} or \code{false}.}
    \begin{enumerate}[label=\texttt{\alph*.}]
      \item The \code{super} keyword allows a class' methods to access attributes and methods of its super-class.
      \item Subclasses can access the \code{private} attributes and methods of their their superclasses.
      \item Subclass constructors must \emph{explicitly} call a superclass constructor if the superclass does not contain a constructor which takes no parameters.
    \end{enumerate}

    \BlueQuestion{Suppose \code{Employee} contains the \code{calculatePay()} method and that \code{Manager} is a subclass of \code{Employee} which overrides \code{calculatePay()}. Consider the following code fragment.}
    \begin{lstlisting}
      public void payIndividual(Employee emp) {
        double pay = emp.calculatePay();
        //  further implementation details not shown
      }
    \end{lstlisting}
    If \code{manager1} is an instance of the \code{Manager} class, which copy of \code{calculatePay()} is evaluated when\\[0pt]
    \code{payIndividual(manager1)} is called? Explain your reasoning.

    \GreenQuestion{Suppose the \code{Pencil} class is a subclass of \code{WritingUtensil} which adds the \code{erase()} method and that \code{pencil1} is an instance of the \code{Pencil} class.} Consider the following code fragment.
    \begin{lstlisting}
      public void removeMistake(WritingUntensil unt) {
        unt.erase();
      }
    \end{lstlisting}

    Explain why the method call, \code{removeMistake(pencil1)} causes an error. How would you fix this error?

    \GreenQuestion{For each pair of classes, indicate whether they should exhibit an ``is-a'' or ``has-a'' relationship. Write ``neither'' if neither relationship appears to work.}
    \begin{enumerate}
      \item \code{Cat}, \code{Animal}
      \item \code{Zoo}, \code{Animal}
      \item \code{Business}, \code{Employee}
      \item \code{Cat}, \code{Dog}
      \item \code{Manager}, \code{Employee}
    \end{enumerate}

    \YellowQuestion{Create classes representing an implementation of the following UML diagram.\\
      {\small\textbf{Note:} Your methods do not need to represent actual working implementations for each class.}
    }
    \begin{center}
      \includegraphics[scale=0.6]{graphics/assignments/apcsa_08_umlDiagram}
    \end{center}

    \pagebreak

    \RedQuestion{Use the \code{Point}, \code{Line}, and \code{Geometry} classes implemented in the previous assignment to complete each of the following tasks.}

    \begin{parts}
        \part Create the \code{Polygon}, \code{Triangle}, and \code{Rectangle} classes with the following specifications.\\
            \code{Polygon}
            \begin{itemize}
              \item Contains an attribute to hold a collection of points.
              \item Contains the appropriate constructor for accepting a collection of points.
              \item Contains the \code{calcPerimeter()} and \code{calcArea()} methods.\\
              {\small\textbf{Note:} Due to the complexity of calculating areas of general polygons, the \code{calcArea()} method can return a default value of $-1$.}
            \end{itemize}

            \code{Triangle}
            \begin{itemize}
              \item A \code{Triangle} ``is-a'' \code{Polygon}.
              \item Overrides the \code{Polygon} constructor in order to verify the correct number of points.
              \item Overrides the \code{calcArea()} method. You can use Heron's Formula below:

              \emph{Given a triangle with side lengths $a$, $b$, and $c$.}
              \[\begin{aligned}
                s &= \dfrac{a + b + c}{2}\\
                A &= \sqrt{s(s - a)(s - b)(s - c)}
              \end{aligned}\]
            \end{itemize}

            \code{Rectangle}
            \begin{itemize}
              \item A \code{Rectangle} ``is-a'' \code{Polygon}.
              \item Overrides the \code{Polygon} constructor in order to verify the correct number of points.
              \item Overrides the \code{calcArea()} method.
            \end{itemize}

      \part Implement a \code{Square} class.\\
      {\small\textbf{Note:} The constructor of your \code{Square} class should verify the points are valid for a square.}

      \part Draw a UML diagram of your entire system of classes.
    \end{parts}
  \end{questions}
\end{document}
