\documentclass[10pt]{exam}
\usepackage[T1]{fontenc}
\usepackage[paper=a4paper,margin=2cm]{geometry}
\usepackage[sfdefault,light]{roboto}

\usepackage[usenames,dvipsnames]{xcolor}
\usepackage{amsmath,amssymb,graphicx,enumitem,listings,multicol,titlesec}
\usepackage{mathtools}

\setlength\parindent{0cm}

\titleformat{\section}{\normalfont\Large\bfseries}{}{0em}{}[{\titlerule[0.5pt]}]
\titleformat{\subsection}{\normalfont\large\bfseries}{}{0em}{}

\newcommand{\floor}[1]{\left\lfloor #1 \right\rfloor}
\newcommand{\code}[1]{\lstinline{#1}}
\newcommand{\blankpage}{\null\thispagestyle{empty}\addtocounter{page}{-1}\newpage}

\lstset{language=Java,
	basicstyle=\ttfamily,
	commentstyle=\color{black!45},
	keywordstyle=\bfseries,
	showstringspaces=false}

\newcommand{\EBox}[2]{
	\colorbox{black!10}{\parbox{0.9875\textwidth}{\bfseries
		#1
	}}

	\colorbox{black!5}{\parbox{0.9875\textwidth}{
		#2
	}}
}

\newcounter{QuestionCounter}
\newcommand{\QBox}[2]{
	\stepcounter{QuestionCounter}
	\colorbox{black!10}{\parbox{0.9875\textwidth}{
		\textbf{Question \#\theQuestionCounter:} #1
	}}

	\colorbox{black!5}{\parbox{0.9875\textwidth}{
		\ \\[#2]
	}}
}

\header{\footnotesize\scshape Lab \#\LabNumber: \LabTitle}{}{}
\cfoot{\footnotesize\scshape \LabCourse\\Woodstock School---Mussoorie, Uttarakhand---India}


%Color-Coded questions
\newcommand{\ColourQuestion}[3]{\renewcommand{\questionlabel}{\colorbox{#1}{\bfseries\color{white}\thequestion}\hfill}\question[#3] #2}
\newcommand{\BlueQuestion}[1]{\ColourQuestion{RoyalBlue}{#1}{3}}
\newcommand{\GreenQuestion}[1]{\ColourQuestion{ForestGreen}{#1}{5}}
\newcommand{\YellowQuestion}[1]{\ColourQuestion{Goldenrod}{#1}{10}}
\newcommand{\RedQuestion}[1]{\ColourQuestion{BrickRed}{#1}{20}}
\newcommand{\PurpleQuestion}[1]{\ColourQuestion{RoyalPurple}{#1}{30}}
\pointsinrightmargin

\header{\footnotesize\scshape Assignment \#\AssignmentNumber: \AssignmentTitle}{}{}
\cfoot{\footnotesize\scshape \AssignmentCourse\\Woodstock School---Mussoorie, Uttarakhand---India}


\def\AssignmentCourse{AP Computer Science A}
\def\AssignmentNumber{01}
\def\AssignmentTitle{Variables \& Operations}

\begin{document}
  \begin{questions}
    \BlueQuestion{Give the value of \code{a} after the execution of each of the following sequences.}
      \begin{parts}
        \part
          \begin{lstlisting}
            int a = 1;
            a = a + a;
            a = a + a;
            a = a + a;
          \end{lstlisting}

        \part
          \begin{lstlisting}
            double a = 2;
            a = a * a;
            a = a * a;
            a = a * a;
          \end{lstlisting}

        \part
          \begin{lstlisting}[]
            boolean a = true;
            a = !a;
            a = !a;
            a = !a;
          \end{lstlisting}
      \end{parts}

    \BlueQuestion{Why does \code{10 / 3} result in the value \code{3} instead of \code{3.33333333}? What modifications would you need to make to ensure the value \code{3.33333333}?}

    \GreenQuestion{What do each of the following print?}
      \begin{parts}
        \part \code{System.out.println(2 + "bc");}
        \part \code{System.out.println(2 + 3 + "bc");}
        \part \code{System.out.println((2 + 3) + "bc");}
        \part \code{System.out.println("bc" + (2 + 3));}
        \part \code{System.out.println("bc" + 2 + 3);}
      \end{parts}

    \GreenQuestion{A physics student gets unexpected results when using the code:}
      \begin{center}
        \code{F = G * mass1 * mass2 / r * r;}
      \end{center}
    to compute values according to the formula $F = Gm_{1}m_{2}/r^{2}$. Explain the problem with the code and indicate how you would fix it.

    \YellowQuestion{\emph{Rolling Dice.} Write a program that generates and prints two random integers between $1$ and $6$ (as if you were rolling dice).}\\[4pt]{\small\textbf{Hint:} You can use \code{Math.random()} to generate a random number. Experiment with its output before deciding how you can use it to restrict your values to the desired results.}

    \RedQuestion{\emph{Loan Payments.} Write a program that calcualtes the monthly payments you would have to make over a given number of years to pay off a loan at a given interest rate, compounded continuously. Given the number of years, $t$, the principal, $P$, and the annual interest rate, $r$, the total amount paid at the end of a loan is given by the formula: $A = Pe^{rt}$.}\\[4pt]{\small\textbf{Hint:} Use \code{Math.exp(n)} to calculate $e^{n}$.}
  \end{questions}
\end{document}
